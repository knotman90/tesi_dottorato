        In this example, a one-dimensional cellular space $R$
    can be used (cf. Figure \ref{fig:cellularspaces}a) to model the
    $10$ grid points $i$ $(i = 0, 1, ...,9)$, while the values that
    the generic cell can assume are simply represented by the $Q = [0,
      100] \subset \mathbb{R}$ substate. By using the forward
    difference formula for the first order derivative and the central
    one for the second order, the following explicit recurrence scheme
    is obtained:
    $$ T_i^{t+1} = T_i^t + k \Delta t \frac{T_{i+1}^t - 2T_i^t +
      T_{i-1}^t}{\Delta x^2}
    $$ denoting that the value of the temperature at the grid point
    $i$ at the time step $t+1$ is a function of the (known) values
    assumed by $T$ at the $i-1$, $i$ and $i+1$ grid points at the
    (previous) time step $t$. Thus, the one dimensional radious-one $X
    = \{-1, 0, 1\}$ neighburhood pattern can be adopted (cf. Figure
    \ref{fig:1Dneighborhood}a) and an elementary process used to
    implement the above recurrence scheme. The explicit formulation of
    the FDM one-dimensional heat conduction model can be therefore
    expressed in terms of the XCA formalism as:

    $$ FDM_{heat} = <R,\Gamma,X,Q,P,\sigma,\gamma>$$

    \noindent where:

    \begin{itemize}

    \item $R = \{0,1,...,9\}$ is the one-dimensional cellular space,
      representing the integer coordinates of the cells of the
      computational domain (cf. Figure \ref{fig:cellularspaces}a).

    \item $\Gamma = \{0, 9\} \subset R$ is the region identifying the
      boundary of the computational domain.


    \item $X = \{-1, 0, 1\}$ is the geometrical pattern defining the
      neighborhood relationship (cf. Figure
      \ref{fig:1Dneighborhood}a).

    \item $Q = [0, 100] \subset \mathbb{R}$ is the set of cell's
      states used to express the temperature values at the grid
      points.

    \item $P = \{k = 1, \Delta t = 0.01, \Delta x = 0.2\}$ is the set
      of the FDM model parameters.
      
    \item $\sigma : Q^3 \rightarrow Q$ is the cell's transition
      function, which implements the explicit recurrence FDM scheme.

    \item $\gamma: Q^{|\Gamma|} \rightarrow Q^{|\Gamma|} \times
      \mathbb{R}$ is the global steering function, used to apply the
      boundary conditions at each computational step.

    \end{itemize}

    The initial conditions of the system are preliminarly defined at
    time $t=0$ by assigning the temperature value 0 to each grid
    point, except for the boundaries where the value 100 is set. The
    global evolution of the system can therefore be obtained by
    applying the $\tau$ global transition function (which in turn
    applies $\sigma$ to each grid point) and eventually the steering
    function $\gamma$ at discete time steps.
    
    
    