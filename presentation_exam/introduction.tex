%%%%%%%%%%%%%%%%%%%%%%%%%%%%%%%%%%%%%
%Presentazione esame di Dottorato Davide Spataro
%introduction.tex
%Purpose: This file contains the introduction to the problem the thesis tries to solve, it shows why single chip performances are not improving as much as the moore's law predicts and introduces the concept of parallelism in computation and in numerical simulation.
%@author Davide Spataro
%@version 1.0 14/01/2018 Eindhoven
%%%%%%%%%%%%%%%%%%%%%%%%%%%%%%%%%%%%

%- Presentati, chi sei + Supervisors e la posizione che hai adesso
%- Posti in cui la tesi è stata sviluppata
%	- calabria + edimburgo + warwick
%	- fai in modo che si evincano le altre cose su cui hai lavorato
%	- velassco e il sistema lineari triangolari	
%
%- La dimensione dei problemi che viene rislta oggigiorno è più grande. Il numero di applicazioni del calcolo scientifico sta aumentando con il numero di devices e di informazioni che la IoT sta fornendo. Questo implica che è richiesta una potenza di calcolo via via sempre maggiore
%
%- La potenza di calcolo aumentava fino a qualche anno fa in maniera esponenziale (mostra grafico) cioè raddoppiando il numero di transistor ogni anno. Ma questo trend non è più sostenibile per una serie di motivi (e ora che lavori in ASML te ne rendi ancora più conto). Dimensione dei transistor e calore per unita di superfice
%
%- Calcolo parallelo diventa fondamentale e negli ultimi 5 anni abbiamo visto un esplosione di parallelismo al punto che anche i telefoni sono macchine parallele con buone prestazioni. Fai esempio dei processori snapdragon.
%
%- Per cosa per esempio queste macchine nell'ambito del calcolo parallelo? risolvono equazioni differenziali etc. 
%Con che metodi lo fanno? molti, ma uno dei più usati è quello delle differenze finite. 
%
%Cosa fa questa tesi? 
 

\section{Motivations} 
\frame{\frametitle{Motivations}
	 

}