\section{Introduction}

\IEEEPARstart{T}{he} beauty and grace of collective coherent
motion ofself-propelled biological organisms namely, flocks of birds,schools of fish, swarms of insects,
slime molds, herds of wilde-beests, has fascinated humans from ancient time his
kind of behavior often referred to asflocking,exists innature at almost every
length scale of observation: from humancrowds, mammalian herds, bird flocks,
fish schools to unicellularorganisms such as amoebae and bacteria, individual
cells, andeven at a microscopic level in the dynamics of actin and
tubulinfilaments and molecular motors. Despite the huge differences inthe scales
of aggregations, the similarities in the patterns thatsuch groups produce have
suggested that general principles mayunderlie collective motion.

An effective approach to study these collective behaviours is represented by
mathematical modelling and numerical simulation as proven by numerous papers
published in this field that are related with both biology and computer
science (vedi citazioni in H. Hamann and T. Schmickl e pure Kishore Dutta).



Simple motion principles of organisms like flocking, shoaling, herding or phenomena
based on random motion can be reflected well enough in mathematical models derived
from physics. In such models, organisms are treated like gas molecules and their motion is
Brownian combined with attraction/repulsion forces. Also ‘mean-field’ approaches, mainly done
in ordinary differential equation (ODE) models, might be useful to model some biological
swarm systems, whenever the assumption of a ‘well mixed’ distribution may be applicable.
However, organisms seldom move really randomly. Nor are they just simple particles. They
pursue specific goals, aggregate or disperse in space, communicate and memorize. They
have specific physiological states (e.g. energy-level) and morphologies (size, weights, etc.).
These factors do not only affect their energetics, they also prominently affect the behaviours
that they (choose to) perform. In addition to that, they frequently interact by direct and indirect
communication and they tend to memorize past effects. Finally, also the environment
in which they operate is highly structured and this heterogeneity is also dynamic.
All these factors describe important discrepancies between biological life forms
and atoms or molecules. Thus, it is likely that models, which were originally derived from physics and chemistry, might not hold well
for biological swarm systems as soon as a certain level of abstraction has to be overcome.
In these cases, individual-based models or even multi-agent
models (\cite{Ferber},\cite{Woolridge}) might be a better choice.

In this paper we present efficient GPGPU implementations of a the
\textsc{ACIADDRI} multi agent flocking model on SIMT architecture (section \ref{sect:aciaddri}
formalize the model and section \ref{sect:gpuimplementation} the parallel
implementation) using the CUDA framework.
Specifically, starting from Reynolds works on bird flocking behavior, \textsc{ACIADDRI} extends it by means of
additional features such as support for multi-species interaction, predator
avoidance, partially observable environment and birds flight constraints
(maximum thrust, stall and peak velocity, etc.).

Section \ref{sect:experiments} reports experiments carried out on the
specific GPU harware and by considering both aggregate motion of a huge number (up to tens of millions) of boids in a virtual
environment and other species or predators avoidance, significant performance
improvement in terms of speedup were obtained ( up to $ 500\times$) while
conclusions and future works are reported at the end of the paper.

