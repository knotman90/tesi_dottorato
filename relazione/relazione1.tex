\part{Attività formative e di ricerca -  \RNum{1} anno}
\section{Corsi e Seminari }
	\begin{tabular}{ p{6cm}  l c }
	\toprule
	\rowcolor{gray!65}	
	\textbf{Denominazione} & \textbf{Docente} & \textbf{Periodo}  \\
		\rowcolor{gray!15}	
	SAT-Based Problem Solving & \textit{Joao Marques-Silva} &  Febbraio, 10-12,
	2015 \\
	  \rowcolor{gray!35}
	Mathematical consequences of the existence of very large 
	infinite cardinal numbers & \textit{Joan Bagaria i Pigrau} &  Marzo, 12,
	2015	 \\
		\rowcolor{gray!15}	
	The Identification and Analysis of
	Genomic Transposable Elements & \textit{John Karro} & Giugno, 5, 2015 \\ 
		\rowcolor{gray!35}	
	 Programming GPUs with CUDA & \textit{Manuel Ujaldon} &
	Giugno, 9-11, 2015 \\  
		\rowcolor{gray!15}	
	Descent, CG and Newton Methods On the Quasi-Wolfe Conditions for Self-Scaling Quasi-Newton Methods  & \textit{Mehiddin Al-Baali} &  Luglio,
	1-2, 2015\\
		\rowcolor{gray!35}	 	   
	Congestion Games   & \textit{Angelo Fanelli} &  Gennaio, 22,
	2015 \\
	  
	\bottomrule
\end{tabular}

\section{Partecipazione a  Conferenze}

\begin{itemize}
	\item PDP-2015- Parallel,Distributed and Network-based Processing - Turku,
	Finlandia - 3-6 Marzo 2015
\end{itemize}


\section{Co-supervisione Tesi}
\begin{itemize}
	\item Laurea Magistrale in Informatica - Rahmat Hidayat - \textit{Multi-agent system with multiple groups modelling of bird flocking on GPGPU}.
\end{itemize}

\section{Articoli e Libri Studiati}
\subsubsection{GPGPU and Fluid Simulation}
\begin{itemize}
	\item  Jos Stam-\textit{Flows on Surfaces of Arbitrary Topology}
	\item Jos Stam, \textit{Stable Fluids}
	\item Martin Guay, Fabrice Colin, Richard Egli. \textit{Simple and Fast
		Fluids. GPU Pro, A.K. Peters,Ltd., 2011, GPU Pro, pp.433-444}
	\item Crane, Llamas, Tariq - GPU Gems 3, Real-Time Simulation and Rendering
	of fluids
	\item Mark Harris - GPU Gems,  Fast Fluid Dynamics Simulation on the GPU
	\item Ali Khajeh-Saeedand J. Blair Perot - Computational FluidDynamics
	Simulations using Many Graphics Processors
	\item Stephen Wolfram - Cellular Automaton Fluids: Basic Theory - Journal
	of Statistical Physics
	\item S. Pratap Vanka, Aaron F. Shinn - Computational Fluid
	Dynamics using GPU: Challenge and Opportunities - Proceedings of the ASME
	2011 International Mechanical Engineering Congress \& Exposition IMECE-2011
	\item Pier Luca Lanzi - Accurate real-time fluid dynamics
	using\textit{ Smoothed-Particle Hydrodynamics} and \textit{CUDA}
	\item Yaroslav D. Sergeyev, Higher order numerical differentiation on the
	Infinity Computer, Optimization Letters, pp 575-585
	\item Yaroslav D. Sergeyev,Solving ordinary differential equations on the
	Infinity Computerby working with infinitesimals numerically, Journal of Applied Mathematics and
	Computation (2013)
	\item Markus Billeter et al. Efficient Stream Compaction on Wide SIMD
	Many-Core Architectures
	\item D.M. Hughes, InK-Compact: In kernel Stream Compaction and Its
	Application to Multi-kernel Data Visualization on GPGPU
\end{itemize}

\subsubsection{Termination Problem - Linear Ranking Function}
\begin{itemize}
	\item Roberto Bagnara, Fred Mesnard - Eventual Linear Ranking Functions 
	\item Jan Leike, Matthias Heizmann - Ranking Templates for Linear Loops
	\item Jan Leike - Ranking Function Synthesis for Linear Lasso Programs
	\item Amir M. Ben-Amram - The Hardness of Finding Linear Ranking Functions for
	Lasso Programs
	\item Amir M. Ben-Amram - Mortality of Iterated Piecewise Affine Functions
	over the Integers:  Decidability and Complexity
	\item Amir M. Ben-Amram, Samir Genaim, Abu Naser Masud - On the Termination of
	Integer Loops
	\item Amir M. Ben-Amram, Samir Genaim - Ranking Functions for
	Linear-Constraint Loops
\end{itemize}

\subsubsection{Programmazione Funzionale ed Haskell}
\begin{itemize}
	\item Michael Barr, Charles Wells - Category Theory for Computing Science
	\item Benjamin C. Pierce - A taste of category theory for computer scientists
	\item Simon Marlow - Parallel and Concurrent Programming in Haskell
	\item Simon Peyton Jones, Satnam Singh - A Tutorial on Parallel and
	Concurrent Programming in Haskell - Lecture Notes from Advanced Functional Programming Summer School 2008
\end{itemize}

\subsection{Pubblicazioni}
\begin{itemize}
	\item Alessio De Rango, Mauizio Macr\'i, Davide Spataro, Donato D'Ambrosio and
	William Spataro, \textbf{Efficient Lava Flows Simulations with OpenCL: A
		preliminary application for Civil Defence Purposes}, \emph{Proceedings of
		The10th International Conference on P2P, Parallel, Grid, Cloud and Internet
		Computing, November 4-6, 2015, Krakow, Poland}
	
	\item Filippone G., Spataro W., D'Ambrosio D., Spataro D., 
	Marocco D.,Trunfio G.A., \textbf{CUDA Dynamic Active Thread List Strategy 
		to Accelerate Debris Flow Simulations} \emph{Proceedings of The 2015 International Conference on Parallel, 
		Distributed and Network-Based Processing (PDP)}, Turku, Finland, pp 330-338.
	
	\item Spataro D., D'Ambrosio D., Filippone G., Spataro W., \textbf{The new
		SCIARA-fv3 numerical model and acceleration by GPGPU strategies}
	\emph{International Journal of High Performance and Applications}.
	
	\item Hidayat R., Spataro D., D’Ambrosio D., Spataro W., \textbf{GPU Ac-
		celerated Modeling of Multi-Agent System with Multiple
		Group for Birds Flocking}, \emph{Parallel, Distributed and Network-
		Based Processing (PDP) 2016, accepted}
	\item Hidayat R., Spataro D., D’Ambrosio D., Spataro W., \textbf{Massive
		Parallel Modeling of Birds Flocking on Multi-Agent Sys-
		tem with Multiple Group}, \emph{IEEE Computing in Science and
		Engineering, Discrete Modeling and Simulation Tools, submitted}
\end{itemize}
\section{Attività Di Ricerca}
Il mio percorso di ricerca segue due filoni principali:
\subsection{Calcolo, Simulazione, Visualizzazione e Programmazione Funzionale
	parallela}
Nel contesto del calcolo parallelo la mia attività di ricerca si occupa di
investigare tecniche ed algoritmi innovativi per la simulazione di fenomeno
naturali attraverso macchine parallele ed in particolare utilizzando la GPGPU
programming. Forte enfasi è data ad una classe ampia di problemi la cui
formulazione matematica è basata su equazioni differenziali e le cui
soluzioni numeriche efficienti sono ottenibili mediante decomposizioni a griglia
(Finite element method - \textbf{FEM} -, Finite Volume Method - \textbf{FVM}
-).



\subsection{Stream Compaction}
Ho studiato e sviluppato  un approccio efficiente al problema della stream compaction su processori
grafici di nuova generazione che hanno introdotto le primitive hardware di
ballotting intra-warps.

La stream compaction/reduction è intuitivamente l'operazione di
rimozione, da una collezione, di elementi che non soddisfano un dato
criterio.

Più nel dettaglio, data una lista di elementi  $A_0..A_{N-1}$ di $N$ elementi
ed un predicato $p:A \mapsto \{True,False\}$, il risultato dell'operazione di
stream compaction di $A$ sotto $p$ è $B=\{x \in A \;|\; p(x)=\ True\}$. 
Spesso, nelle applicazioni reali è sufficiente riordinare $A$ tale che gli
elementi validi siano raggruppati. 

La stream compaction è parte fondamentale dell'implementazione
efficiente di molti algoritmi paralleli dove strutture dati sparse di grandi
dimensioni possono essere  difficili da processare, come ad esempio  in
algoritmi di parallel breadht tree traversing, ray tracing, AC, etc.,
e possono portare ad un degrado prestazionale in temini di tempi
d'esecuzione ed efficienza parallela\footnote{Definita come $E=\frac{S}{N}
	$ con $S=\frac{T_1}{T_N}$ ed $N$ il numero di processori utilizzati.}.
L'implementazione seriale è banale mentre quella parallela richiede uno sforzo maggiore, specialmente su
architettura SIMT (single instruction multiple thread) dove è richiesto un
ulteriore step intermedio, la prefix sum (o scan). 
\begin{mydef}{Prefix Scan}
	
	Dato un operatore binario associativo  $\diamond$, una collezione di $N$
	elementi $V_{0 \ldots {N-1}}$ ed un elemento identità $I$ per $\diamond$
	allora la prefix scan è la collezione $P$ definita come segue:
	\[
	P= \{I,V_0,(V_0 \diamond V_1),(V_0 \diamond V_1 \diamond V_2) \ldots
	(V_0 \diamond V_1 \diamond \ldots \diamond V_{N-1}) \}
	\]
	
\end{mydef}

\subsubsection{Stream Compaction parallela}\
Supponiamo che $P$ sia il numero di processori ed $N$ la size della
collezione, $N>P$ (nei casi reali $N>>P$) e che l'input stream
sia diviso in substream di size $S$.
L'algoritmo è composto da tre fasi distinte:
\begin{enumerate}
	\item Ogni processore $p_i$ conta in modo indipendente il numero di
	elementi validi nel proprio substream salvandolo in $PC[p_i]$
	\item Viene effettuata un' operazione di prefix sum ($\diamond = +, I=1$)
	su $PC$ producendo l'offset a cui ogni processore scriverà il proprio
	risultato finale $PO[p_i]$.
	\item Ogni processore scrive in modo indipendente il proprio output (gli
	elementi validi) senza interferire con gli altri nella locazione
	$OUT[PO[p_i]]$.
\end{enumerate}


\subsubsection{Stream Compaction on SIMT Hardware}\label{smparallel}
L'hardware  delle moderne GPU è composto di un numero dell'ordine delle decine
di streaming multiprocessors (SMM), che intuitivamente sono un array di processori SIMD a
loro volta composti dalle SIMD lanes, i cosidetti Streaming Processors. Le
nuove architetture hardware NVIDIA contano 16 SMM e 128 SP per SMM per un
totale di oltre 2000 cores.
Ogni SMM esegue il threads a blocchi indivisibili di 32, i warp, all'interno
dei quali ogni thread esege in modo sincrono le stesse istruzioni. SMM non
sono processori scalari e quindi usando lo stesso approccio descritto al
paragrafo \ref{smparallel}, si otterebbero valori di efficienza parallelo
molto bassi a causa del fatto che per ogni SMM verrebbe utilizzato un solo
core (avremmo infatti un altissimo numero di idle SIMD
lanes,$ 16\times(128-1)=2048-16$).

Per ovviare a questo problema ogni fase dell'algoritmo descritto al paragrafo
\ref{smparallel} è stata riadattata per sfruttare al massimo la potenza di
calcolo disponibile, ed in particolare:
\begin{description}
	\item[Phase 1]
	Ogni SMM ha il compito di gestire una porzione dello stream di input, e
	procedendo a gruppi di \textbf{warpSize} calcola, tramite un operazione di
	reduce il numero di elementi validi nella
	porzione dell'input gestita dal SMM.
	\item[Phase 2]
	Output della fase precendente è un array di counter che viene usato durante
	questa fase per calcolare, attraverso l'operazione di prefix sum, l'offset
	di output per ogni SMM.
	\item[Phase 3]
	Questa fase prende in input lo stream da compattare e l'array di offset
	calcolato alla fase precedente e restituisce in output lo stream compattato.
	Utilizza le funzioni di ballotting (introdotte nelle moderne architetture
	GPU) che consentono a i thread di uno stesso warp di comunicare senza
	incorrere in alcun overhead. Intuitivamente computa warp, block a grid
	offset, in modo che ogni thread abbia a disposizione l'esatta locazione di
	output per un elemento valido dello stream di input\footnote{Che ha indice
		di output: offset del thread nel warp sommato a quello del warp nel blocco,
		e quello del blocco nella griglia.}.
\end{description}
I moduli CUDA/C che implementano questa operazione sono liberamente
disponibili al seguente url: \url{https://github.com/knotman90/cuStreamComp}


\subsection{Bird Flock Simulation on GPU}
Ho lavorato ad un simulatore parallelo su GPU per la simulazione di Bird
flocks basato sul modello di proposto da Craig W. Reynolds in 1987. Esso
descrive tre leggi cui ogni agente della simulazione ad ogni step
temporale obbedisce:
\begin{description}
	\item[1: coesione] è il vettore forza diretto verso il centro di massa dello
	stormo 
	\item [2: separazione]  è il vettore forza, che mantiene l'agente ad una
	distanza minima da tutti gli altri boids
	\item [3: allineamento] è la forza che sincronizza i vettori velocità di un
	agente con i suoi vicini
\end{description}

Il sistema, adotta un approccio multi-agents multiple
groups per la modellazione della dinamica degli stormi. Il modello sviluppato
è basato su quello di Reynolds, e lo estende aggiungendo il supporto alla
coesistenza e l'interazione di diverse specie e per la presenza di predatori.
Gli esperimenti condotti mostrano che l'utilizzo della GPGPU in questo ambito ha
garantito un significativo aumento delle performance in termini di speedup
(fino a 700x) confermando la validità di questa tecnologia anche in ambito
multi-agents modeling. 

Questo lavoro ha ispirato il design di una libreria per
agenti autonomi in GPU che verrà sviluppata durante il secondo anno di dottorato.

\subsection{Soluzioni numeriche di ODE e PDE utilizzando Grossone}
Assieme al prof. Ya. D. Sergeyev, basandoci su un suo lavoro che applica il
sistema numerico Grossone, da lui ideato, alla risoluzione di ODE e PDE
abbiamo sviluppato una libreria Haskell per la manipolazione numerica di
equazioni differenziali nel suddetto sistema numerico. Questa sarà uno dei
blocchi costituenti di un solver parallelo a precisione arbitraria.

 \subsection{Termination problem e ranking functions}
L'analisi della terminazione è stata oggetto di numerosi studi negli ultimi
anni, con un conseguente aumento del numero e dell'accuratezza dei tools per
lo studio statico della terminazione di programmi.

Il caso generale, l'halting problem, è indecidibile, ma molte sottoclassi
sono decidibili e ampiamente studiate, come ad esempio quello della
terminazione dei cosidetti loop lineari deterministici. In questo contesto la mia ricerca si occupa di
investigare il ruolo e le proprietà di una particolare classe di funzioni, le
\textbf{multiphase linear ranking functions}, che si sono dimostrate essere un
valido strumento per la dimostrazione della terminazione dei suddetti loop (
si vedano definizioni seguenti). La mia ricerca è orientata all'investigazione di
problemi teorici riguardanti:
\begin{itemize}
	\item L'esistenza di un algoritmo (assieme a soundness e completeness) per
	la sintetizzazione di questo tipo di ranking function sia sui razionali che sugli interi.
	\item L'esistenza di un bound di qualche tipo sul numero di componenti.
	\item Individuazione di sottoclassi di SLC che hanno una RF di questo tipo.
\end{itemize}



\begin{mydef}
	Un single path linear constraint loop  su $n$ variabili $x_1 \ldots x_n$ \`e
	della forma
	
	\[
	while (Bx \leq b); \:\: do \: \;A \begin{pmatrix}x\\x'\end{pmatrix} \leq c
	\] 
	dove $B \in \mathbb{Q}^{m \times n}$, $x,x' \in \mathbb{Q}^n$, $b \in
	\mathbb{Q}^n$, $A \in \mathbb{Q}^{p \times n}$ e $c \in
	\mathbb{Q}^p$. \\ $(Bx \leq b)$ e $A \begin{pmatrix}x\\x'\end{pmatrix} \leq c$
	sono chiamati rispettivamente \textbf{guard} e \textbf{update} del loop.
	Quest'ultimo è \textbf{deterministico} se pu\`o essere riscritto come un sistema
	di uguaglianze ($A \begin{pmatrix}x\\x'\end{pmatrix} = c$).
	Il vettore $\begin{pmatrix}x\\x'\end{pmatrix} \in \mathbb{Q}^{2n}$ vive nello
	spazio delle transizioni del loop. Pi\`u formalmente 
	$\begin{pmatrix}x\\x'\end{pmatrix}$ è una transizione se $x$  soddisfa il guard 
	e $x'$ l' update.
\end{mydef}

\subsubsection{Multiphase ranking function - MPhiRF}

Consideriamo un SLC Q. Una tupla \(<\rho_1,...,\rho_d>\) è una MPHiRF
se per ogni transizione \(\begin{pmatrix}x\\x'\end{pmatrix} \in Q\), \(\exists i \in [1..d]\) tale che:
\begin{enumerate}
	\item $\begin{aligned}[t]
	\rho_j(x)-\rho_j(x') \geq 1, \;\;\forall j \leq i\\
	\end{aligned}$
	\item $\begin{aligned}[t]
	\rho_i(x) \geq 0\\
	\end{aligned}$
	\item $\begin{aligned}[t]
	\rho_j(x)< 0,  \;\;\forall j<i\\
	\end{aligned}$
\end{enumerate}

Intuitivamente esiste una funzione $\rho_1(x)$ che è sempre
decrescente in  Q e quando diventa negativa $\rho_2(x)$ inizia a decrescere,
e quando anch'essa diventa negativa ne esiste un altra che decresce nella stessa
maniera e così via.

Ovviamente tutto ciò \textbf{implica la terminazione del loop}.

Il punto (1) significa che quando una componente inizia a decrescere, lo farà
per sempre.
I punti (2,3) significano che se usiamo la $i$-esima componente per il ranking
di $\begin{pmatrix}x\\x'\end{pmatrix}$, allora tutte le componenti $j<i$ sono
già negative, altrimenti ne avremmo usata una tra quelle per effettuare il ranking di $\begin{pmatrix}x\\x'\end{pmatrix}$
-- perchè sono anch'esse decrescenti.
Si noti come (3) è in qualche modo ridondante perchè avremmo potuto definire la
stessa classe di funzioni prendendo il più piccolo indice $i$ tale che (1,2)
fossero verificati.

Come esempio, consideriamo il seguente loop:
\[while ( x>0 )\: { x'=x+y, y'=y-1 }\]

che ha MPhiRF: \(<y,x>.\)
Tutte le transizioni del tipo $(x,y\geq 0)$ hanno come ranking function
$\rho_1(x)=y$ che è positiva e decrescente $(y-(y-1)) =1 \geq 1$. Quando essa
diventa negative --per le transizioni del tipo $(x,y<0)$--, allora
$\rho_2(x)=x$ è positiva e decrescente $(x - (x-y)) = y \geq 1$.
